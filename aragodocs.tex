% !TEX TS-program = pdflatex
% !TEX encoding = UTF-8 Unicode

% This is a simple template for a LaTeX document using the "article" class.
% See "book", "report", "letter" for other types of document.

\documentclass[12pt]{article} % use larger type; default would be 10pt
\usepackage[fontsize=12pt]{fontsize}
\renewcommand{\baselinestretch}{1.5} % linespacing=1.5
\usepackage[T2A]{fontenc} % кодировка
\usepackage{fontspec} % for changing fonts 
% \setmainfont{Times New Roman}
\setmainfont{Minion Pro}
% \setmainfont{Roboto}
% \usepackage[utf8]{inputenc} % set input encoding (not needed with XeLaTeX)
\usepackage[english]{babel} % for cyrillic letters support
\usepackage[parfill]{parskip} % remove space vefore parapraph
\setlength{\parskip}{0pt} % additional skip between paragraphs


%%% PAGE DIMENSIONS
\usepackage{geometry} % to change the page dimensions
\geometry{a4paper} % or letterpaper (US) or a5paper or....
\geometry{left=30mm,right=20mm,top=15mm, bottom=15mm} % for example, change the margins to 2 inches all round
% \graphicspath{{./images/}}
% \geometry{landscape} % set up the page for landscape
%   read geometry.pdf for detailed page layout information

\usepackage{graphicx} % support the \includegraphics command and options

% \usepackage[parfill]{parskip} % Activate to begin paragraphs with an empty line rather than an indent

%%% PACKAGES
\usepackage{blindtext}
\usepackage{booktabs} % for much better looking tables
\usepackage{multirow}
\usepackage{array} % for better arrays (eg matrices) in maths
\usepackage{paralist} % very flexible & customisable lists (eg. enumerate/itemize, etc.)
\usepackage{verbatim} % adds environment for commenting out blocks of text & for better verbatim
\usepackage{subfig} % make it possible to include more than one captioned figure/table in a single float
\usepackage{cmap} % search in pdf
\usepackage{longtable} % for long tables 
\usepackage{lscape}
\usepackage{hyperref} % for hyperlinks
\usepackage{listings} % for code listings
% These packages are all incorporated in the memoir class to one degree or another...

%% Useful packages
\usepackage[colorinlistoftodos]{todonotes}
% \usepackage[colorlinks=true, allcolors=blue]{hyperref}



\usepackage{amsmath,amsthm,amsfonts,amssymb,amscd, fancyhdr, color, comment, graphicx, environ}
\usepackage{float}
\usepackage{mathrsfs}
\usepackage[math-style=ISO]{unicode-math}
\setmathfont{TeX Gyre Termes Math}
\setmonofont{Hack Nerd Font Mono}
\usepackage{lastpage}
\usepackage[dvipsnames]{xcolor}
% \usepackage[framemethod=TikZ]{mdframed}
\usepackage{indentfirst}
\usepackage{thmtools}
\usepackage{shadethm}
\usepackage{setspace}


\definecolor{codegreen}{rgb}{0,0.6,0}
\definecolor{codegray}{rgb}{0.5,0.5,0.5}
\definecolor{codepurple}{rgb}{0.58,0,0.82}
\definecolor{backcolour}{rgb}{0.95,0.95,0.92}


%%% HEADERS & FOOTERS
\usepackage{fancyhdr} % This should be set AFTER setting up the page geometry
\pagestyle{fancy} % options: empty , plain , fancy
\renewcommand{\headrulewidth}{0pt} % customise the layout...
\lhead{}\chead{}\rhead{}
\lfoot{}\cfoot{\thepage}\rfoot{}

%%% SECTION TITLE APPEARANCE
\usepackage{titlesec}
\usepackage{sectsty}
\sectionfont{\centering}
\subsectionfont{\centering}

\usepackage{enumitem}
\setlist{nolistsep}

\hypersetup{
    colorlinks=true,
    linkcolor=black,
    filecolor=magenta,      
    urlcolor=blue,
    pdftitle={Overleaf Example},
    pdfpagemode=FullScreen,
    }
\urlstyle{same}




\lstdefinestyle{mystyle}{
    extendedchars=\true,
    inputencoding=utf8x,
    backgroundcolor=\color{backcolour},   
    commentstyle=\color{codegreen},
    keywordstyle=\color{magenta},
    numberstyle=\tiny\color{codegray},
    stringstyle=\color{codepurple},
    basicstyle=\ttfamily\footnotesize,
    breakatwhitespace=false,         
    breaklines=true,                 
    captionpos=b,                    
    keepspaces=true,                 
    numbers=left,                    
    numbersep=5pt,                  
    showspaces=false,                
    showstringspaces=false,
    showtabs=false,                  
    tabsize=2
}

\lstset{style=mystyle}




% \allsectionsfont{\sffamily\mdseries\upshape} % (See the fntguide.pdf for font help)
% (This matches ConTeXt defaults)

%%% ToC (table of contents) APPEARANCE
% \usepackage[nottoc,notlof,notlot]{tocbibind} % Put the bibliography in the ToC
% \usepackage[titles,subfigure]{tocloft} % Alter the style of the Table of Contents
% \renewcommand{\cftsecfont}{\rmfamily\mdseries\upshape}
% \renewcommand{\cftsecpagefont}{\rmfamily\mdseries\upshape} % No bold!
\makeatletter
\renewcommand{\fnum@figure}{Picture. \thefigure}
\makeatother



\renewcommand\lstlistingname{Algorithm}
\renewcommand\lstlistlistingname{Algorithms}
\def\lstlistingautorefname{Alg.}

\title{Market place Course Work}

\author{Baktybekov.N, Manurov.A, Erkin}

\begin{document}

\maketitle

\newpage
\tableofcontents{}
\setcounter{page}{1}

\newpage
\addcontentsline{toc}{section}{Introduction}
\section*{Введение}

\textbf{Ввод в экскурс}


Маркетплейсы в привычном для нас понимании существовали практически всю историю с момента зарождения человечества. С развитием науки и техники маркетплейсы также развивались становясь все удобнее и доступнее.
С изобретением компьютера и последующей компьютеризацией маркетплейсы трансформировались и перебрались в интернет пространство.
На данный момент существует тысячи интернет платформ маркетплейсов предоставляющих доступные способы размещения и просматривания товаров и услуг таких как Avito, Lalafo, Bazar.kg, Kerek.kg  и т.д. 

Цель этого проекта - создание маркетплейса, аналогичного Avito, используя язык программирования Java и фреймворк Spring. В этой документации будет обзор существующих решений на рынке, функции, которыми должен обладать сайт, и то, чем он может быть лучше существующих решений.
Обзор существующих решений на рынке



\section{Исследование предметной области}
\textbf{Обзор существующих решений на рынке}

На рынке существует множество маркетплейсов, таких как Amazon, eBay, Etsy и Avito. Каждый из них имеет свои функции и особенности.

Amazon - это один из крупнейших онлайн-магазинов, который предлагает широкий ассортимент товаров. Он имеет функции, такие как рекомендации, быстрая доставка и отзывы покупателей.

eBay - это онлайн-аукцион, который позволяет пользователям продавать и покупать товары. Он имеет функции, такие как аукционные лоты, функцию "Купить сейчас", а также возможность общения между покупателями и продавцами.

Etsy - это маркетплейс, который специализируется на продаже ручной работы, винтажных товаров и материалов для творчества. Он имеет функции, такие как персональные рекомендации, возможность создания своего магазина и сообщества покупателей.

Avito - это крупнейший российский маркетплейс, который предлагает широкий ассортимент товаров, услуг и недвижимости. Он имеет функции, такие как поиск по регионам, возможность размещения бесплатных объявлений и множество категорий товаров.
Функции, которыми должен обладать сайт


\textbf{Функции нашего сайта}
Для создания маркетплейса, подобного Avito, необходимо реализовать следующие функции:

    Регистрация и аутентификация пользователей
    Создание профилей пользователей
    Размещение объявлений и управление ими
    Поиск товаров и услуг по категориям и регионам
    Отображение информации о товарах и услугах
    Возможность связаться с продавцом через сайт
    Оставление отзывов и рейтинга продавцов
    Настраиваемые уведомления



\textbf{Цель:}

Конечная цель проекта - создание функционального маркетплейса, который позволит пользователям размещать свои объявления о продаже товаров и услуг, а также искать и приобретать нужные им товары и услуги у других пользователей.

Этот проект позволит студенту лучше понять язык программирования Java и фреймворк Spring, а также научиться проектировать и разрабатывать веб-приложения. Создание маркетплейса также требует понимания основных принципов взаимодействия пользователей, управления данными и безопасности веб-приложений.

Поэтому, целью этого проекта является не только создание конечного продукта, но и получение опыта разработки веб-приложения, понимание принципов его работы и применение на практике основных знаний, полученных на предыдущих этапах обучения.









\end{document}
