\input{Preambule_course_work.tex}
\title{Market place Course Work}

\author{Baktybekov.N, Manurov.A, Erkin}

\begin{document}

\maketitle

\newpage
\tableofcontents{}
\setcounter{page}{1}

\newpage
\addcontentsline{toc}{section}{Introduction}
\section*{Введение}

\textbf{Ввод в экскурс}


Маркетплейсы в привычном для нас понимании существовали практически всю историю с момента зарождения человечества. С развитием науки и техники маркетплейсы также развивались становясь все удобнее и доступнее.
С изобретением компьютера и последующей компьютеризацией маркетплейсы трансформировались и перебрались в интернет пространство.
На данный момент существует тысячи интернет платформ маркетплейсов предоставляющих доступные способы размещения и просматривания товаров и услуг таких как Avito, Lalafo, Bazar.kg, Kerek.kg  и т.д. 

Цель этого проекта - создание маркетплейса, аналогичного Avito, используя язык программирования Java и фреймворк Spring. В этой документации будет обзор существующих решений на рынке, функции, которыми должен обладать сайт, и то, чем он может быть лучше существующих решений.
Обзор существующих решений на рынке



\section{Исследование предметной области}
\textbf{Обзор существующих решений на рынке}

На рынке существует множество маркетплейсов, таких как Amazon, eBay, Etsy и Avito. Каждый из них имеет свои функции и особенности.

Amazon - это один из крупнейших онлайн-магазинов, который предлагает широкий ассортимент товаров. Он имеет функции, такие как рекомендации, быстрая доставка и отзывы покупателей.

eBay - это онлайн-аукцион, который позволяет пользователям продавать и покупать товары. Он имеет функции, такие как аукционные лоты, функцию "Купить сейчас", а также возможность общения между покупателями и продавцами.

Etsy - это маркетплейс, который специализируется на продаже ручной работы, винтажных товаров и материалов для творчества. Он имеет функции, такие как персональные рекомендации, возможность создания своего магазина и сообщества покупателей.

Avito - это крупнейший российский маркетплейс, который предлагает широкий ассортимент товаров, услуг и недвижимости. Он имеет функции, такие как поиск по регионам, возможность размещения бесплатных объявлений и множество категорий товаров.
Функции, которыми должен обладать сайт


\textbf{Функции нашего сайта}
Для создания маркетплейса, подобного Avito, необходимо реализовать следующие функции:

    Регистрация и аутентификация пользователей
    Создание профилей пользователей
    Размещение объявлений и управление ими
    Поиск товаров и услуг по категориям и регионам
    Отображение информации о товарах и услугах
    Возможность связаться с продавцом через сайт
    Оставление отзывов и рейтинга продавцов
    Настраиваемые уведомления



\textbf{Цель:}

Конечная цель проекта - создание функционального маркетплейса, который позволит пользователям размещать свои объявления о продаже товаров и услуг, а также искать и приобретать нужные им товары и услуги у других пользователей.

Этот проект позволит студенту лучше понять язык программирования Java и фреймворк Spring, а также научиться проектировать и разрабатывать веб-приложения. Создание маркетплейса также требует понимания основных принципов взаимодействия пользователей, управления данными и безопасности веб-приложений.

Поэтому, целью этого проекта является не только создание конечного продукта, но и получение опыта разработки веб-приложения, понимание принципов его работы и применение на практике основных знаний, полученных на предыдущих этапах обучения.









\end{document}
