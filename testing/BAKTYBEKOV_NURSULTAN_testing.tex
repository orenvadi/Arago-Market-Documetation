\input{Preambule_course_work.tex}

\title{Тестрование Авторизции и регистрации}
\author{Бактыбеков Н.Б.}

\begin{document}
\input{title/title.tex}

% \maketitle

\newpage
% \tableofcontents{}
\setcounter{page}{1}

% \newpage
\addcontentsline{toc}{section}{Проверка авторизации и регистрации}


\section*{Проверка авторизации и регистрации}


\textbf{Введение}

Авторизационное и регистрационное тестирование — это тип тестирования программного обеспечения, который гарантирует, что пользователи могут получить доступ только к тем ресурсам и функциям, на доступ к которым им предоставлен доступ. Этот тип тестирования важен для обеспечения безопасности и конфиденциальности данных пользователей.

\textbf{Цели тестирования}

Целями авторизационного и регистрационного тестирования являются:

\begin{compactitem}
    \item    Уьбедиться, что пользователи могут получить доступ только к тем ресурсам и функциям, к которым у них есть доступ.
    \item    Уьбедиться, что пользователи не могут получить доступ к ресурсам и функциям, на доступ к которым у них нет прав.
    \item    Уьбедиться, что пользователи не могут обойти процесс авторизации.
\end{compactitem}

\textbf{Методы тестирования}


Существует множество методов, которые можно использовать для проверки авторизации и регистрации. Некоторые из наиболее распространенных методов включают в себя:

\begin{compactitem}

    \item    Тестирование «черного ящика»: при тестировании «черного ящика» тестер не имеет никаких знаний о внутренней реализации программного обеспечения. Тестер просто вводит данные в программу и наблюдает за выводом.
    \item    Тестирование белого ящика: при тестировании белого ящика тестер знает внутреннюю реализацию программного обеспечения. Эти знания можно использовать для выявления потенциальных уязвимостей безопасности.
    \item    Тестирование безопасности. Тестирование безопасности — это тип тестирования, направленный на выявление и устранение уязвимостей безопасности. Тестирование безопасности может использоваться для выявления уязвимостей в процессе авторизации и регистрации.
    
\end{compactitem}

\textbf{Инструменты тестирования}

Существует множество инструментов, которые можно использовать для автоматизации проверки авторизации и регистрации. Некоторые из самых популярных инструментов включают в себя:

\begin{compactitem}

    \item    OWASP Zed Attack Proxy (ZAP): ZAP — это бесплатный инструмент тестирования безопасности с открытым исходным кодом, который можно использовать для выявления и устранения уязвимостей безопасности.
    \item    Burp Suite: Burp Suite — это коммерческий инструмент для тестирования безопасности, который предлагает широкий спектр функций для выявления и устранения уязвимостей безопасности.
    \item    Nessus: Nessus — это коммерческий сканер уязвимостей, который можно использовать для выявления уязвимостей безопасности в различных системах.

\end{compactitem}

\textbf{Контрольный список тестирования}

Ниже приведен контрольный список пунктов, которые следует учитывать при тестировании авторизации и регистрации:

\begin{compactitem}

    \item    Уьбедиться, что пользователи могут получить доступ только к тем ресурсам и функциям, на доступ к которым они имеют право.
    \item    Уьбедиться, что пользователи не могут получить доступ к ресурсам и функциям, на доступ к которым у них нет прав.
    \item    Уьбедиться, что пользователи не могут обойти процесс авторизации.
    \item    Уьбедиться, что процесс авторизации безопасен и его нельзя легко обойти.
    \item    Уьбедиться, что процесс авторизации является масштабируемым и может обрабатывать большое количество пользователей.
    \item    Уьбедиться, что процесс авторизации надежен и не дает сбоев под нагрузкой.

\end{compactitem}

\textbf{Заключение}

Тестирование авторизации и регистрации является важной частью тестирования программного обеспечения. Тестируя процесс авторизации и регистрации, тестировщики могут помочь обеспечить безопасность и конфиденциальность данных пользователей.



\end{document}
