\input{Preambule_course_work.tex}
\title{Market place Course Work}

\author{Baktybekov.N, Manurov.A, Erkin}

\begin{document}

\maketitle

\newpage
\tableofcontents{}
\setcounter{page}{1}

\newpage
\addcontentsline{toc}{section}{Введение}

Введение.
% 1. Основание для разработки.
% 2. Назначение разработки.
% 3. Требования к программе или программному
% изделию.
% 4. Требования к программной документации.
% 5. Технико-экономическое обоснование.
% 6. Стадии и этапы разработки.
% 7. Порядок контроля и приемки.
% 8. Приложения.

\section*{Введение}


Введение:

Данное техническое задание разработано для создания программного продукта - веб-приложения для онлайн-бронирования и оплаты услуг в сфере туризма. Основанием для разработки является необходимость упрощения процесса бронирования и оплаты туристических услуг, а также повышения удобства и доступности для клиентов.

Назначение данной разработки заключается в создании функционального и удобного веб-приложения, которое позволит пользователям легко и быстро выбирать и оплачивать туристические услуги, а также отслеживать свои бронирования и получать информацию о своих поездках.

Для достижения поставленной цели необходимо выполнение следующих требований к программе: создание удобного и интуитивно понятного интерфейса, обеспечение безопасности персональных данных пользователей, поддержка различных форм оплаты, интеграция с сервисами поиска туристических услуг и обеспечение высокой производительности и доступности приложения.

Техническое задание также устанавливает требования к программной документации, а также включает технико-экономическое обоснование, стадии и этапы разработки, порядок контроля и приемки, а также приложения, необходимые для успешной реализации проекта.









\newpage
\section{Основание для разработки.}



Основанием для разработки проекта является потребность нашей компании в создании электронной платформы для продажи товаров в Интернете. В настоящее время все больше потребителей предпочитают совершать покупки в Интернете, что делает разработку собственного интернет-магазина необходимым шагом для нашей компании в укреплении позиций на рынке и увеличении объемов продаж. Кроме того, создание собственного интернет-магазина позволит нам расширить географию нашей деятельности и увеличить количество клиентов, что в свою очередь приведет к увеличению прибыли.








\newpage
\section{Назначение разработки.}



Назначение разработки данного проекта заключается в создании программного обеспечения, которое будет обеспечивать удобство и комфорт для пользователей при покупке товаров в интернет-магазине. Разработка программы должна способствовать увеличению продаж, удобству навигации по сайту и повышению уровня сервиса. Также важным назначением разработки является создание функционального и надежного программного продукта, который будет соответствовать всем требованиям заказчика и пользователей.







\newpage
\section{Требования к программе или программному изделию.}





Требования к программе или программному изделию:

    Функциональность:
    1.1. Возможность регистрации пользователей и создания личного кабинета;
    1.2. Возможность добавления, редактирования и удаления товаров и услуг;
    1.3. Возможность просмотра информации о товарах и услугах, включая фотографии и описания;
    1.4. Возможность оформления заказа и оплаты товаров и услуг;
    1.5. Возможность просмотра истории заказов и состояния текущих заказов;
    1.6. Возможность обратной связи с администрацией сайта.

    Безопасность:
    2.1. Защита личных данных пользователей;
    2.2. Защита от взлома и хакерских атак;
    2.3. Защита от SQL-инъекций и XSS-атак.

    Скорость и производительность:
    3.1. Быстрая загрузка страниц сайта;
    3.2. Высокая скорость обработки запросов;
    3.3. Поддержка большого количества пользователей одновременно.

    Кросс-платформенность:
    4.1. Совместимость с различными браузерами и операционными системами;
    4.2. Адаптивный дизайн для разных разрешений экранов, включая мобильные устройства, планшеты и настольные компьютеры.

    Надежность:
    5.1. Бесперебойная работа в течение всего времени доступности для пользователей;
    5.2. Предусмотрен механизм аварийного восстановления и быстрой реакции на проблемы.

    Удобство использования:
    6.1. Простой и понятный интерфейс;
    6.2. Легкий поиск и покупка товаров;
    6.3. Современный дизайн, соответствующий последним трендам в веб-дизайне.

    Поддержка:
    7.1. Предусмотрен механизм технической поддержки для пользователей, позволяющий быстро решать проблемы и отвечать на вопросы;
    7.2. Поддержка может быть реализована через электронную почту, онлайн-чат или телефон.

    Совместимость:
    8.1. Совместимость с различными браузерами и операционными системами









\newpage
\section{Требования к программной документации.}




Требования к программной документации для данного проекта включают следующие пункты:

    Техническое описание - должно содержать подробное описание функциональности, архитектуры и основных технологий, используемых в проекте.

    Руководство пользователя - должно содержать подробное описание того, как пользоваться программным продуктом, включая инструкции по установке и настройке.

    Руководство системного администратора - должно содержать описание процедур установки, настройки, администрирования и мониторинга программного продукта.

    Документация по API - должна содержать описание интерфейсов программного продукта, которые могут использоваться другими приложениями для интеграции.

    Исходный код - должен быть документирован и комментирован для облегчения понимания его работы и изменения в будущем.

    Тестовая документация - должна содержать описание тестовых сценариев, результаты тестирования и инструкции по проведению тестирования.

    Документация по обслуживанию и поддержке - должна содержать инструкции по устранению возможных проблем и ответы на часто задаваемые вопросы.

Все документы должны быть написаны на понятном языке, соответствовать установленным стандартам и обеспечивать достаточный уровень информации для успешной эксплуатации и поддержки программного продукта.









\newpage
\section{Технико-экономическое обоснование.}






Технико-экономическое обоснование проекта является необходимым для определения его целесообразности и экономической эффективности. Наш проект представляет собой разработку программного обеспечения для автоматизации бизнес-процессов в компании.

Предполагается, что внедрение программы позволит значительно сократить затраты на ручной труд, сократить время на выполнение задач и увеличить эффективность работы компании в целом. Кроме того, автоматизация бизнес-процессов повысит качество работы и сократит вероятность ошибок.

Для реализации проекта необходимо закупить оборудование и программное обеспечение, а также оплатить труд программистов и тестировщиков. По результатам проведенных расчетов, общая стоимость проекта составляет X тысяч долларов США.

Ожидаемый доход от реализации проекта составляет Y тысяч долларов США. Окупаемость проекта составит Z месяцев. В перспективе, внедрение программы позволит значительно сократить затраты на ручной труд и увеличить доходы компании.

Таким образом, технико-экономическое обоснование проекта показывает его целесообразность и экономическую эффективность, а также является основой для принятия решения о дальнейшей реализации проекта.







\newpage
\section{Стадии и этапы разработки.}






Стадии и этапы разработки проекта должны включать следующие этапы:

    Планирование проекта:

    Определение целей и задач проекта;
    Составление плана работ и графика выполнения;
    Определение необходимых ресурсов.

    Анализ и проектирование:

    Сбор и анализ требований к программе;
    Проектирование архитектуры программного продукта;
    Создание технического задания на разработку программы.

    Разработка:

    Создание программного кода;
    Тестирование программного продукта;
    Устранение обнаруженных ошибок и недоработок.

    Внедрение:

    Подготовка к установке программы на рабочие места;
    Установка программы на рабочие места;
    Настройка программы и ввод данных.

    Эксплуатация и поддержка:

    Поддержка пользователей;
    Проведение работ по сопровождению программы;
    Внесение изменений и улучшений в программный продукт.

Каждый этап должен быть осуществлен с соблюдением определенных этапов и процедур, и все этапы должны быть завершены до сдачи программного продукта заказчику.






\newpage
\section{Порядок контроля и приемки Приложения.}







Порядок контроля и приемки включает в себя следующие шаги:

    Предварительная проверка – на этом этапе проверяются программные модули и компоненты на соответствие требованиям, а также на возможные ошибки и неполадки.

    Тестирование – проводятся функциональные и интеграционные тесты, которые позволяют проверить работу программного продукта в различных условиях и на различных платформах.

    Оценка качества – на этом этапе оценивается качество программного продукта по ряду критериев, таких как функциональность, удобство использования, быстродействие и т.д.

    Приемка – после успешного прохождения всех проверок и тестов, производится приемка программного продукта. По результатам приемки составляется акт, в котором указывается, что продукт соответствует требованиям технического задания и может быть введен в эксплуатацию.

    Гарантийное обслуживание – после ввода в эксплуатацию, проводится гарантийное обслуживание программного продукта, которое включает в себя устранение возможных ошибок и неполадок, а также обновление продукта до новых версий и выпусков.

    Поддержка – после окончания гарантийного обслуживания, предоставляется услуга по поддержке программного продукта, которая включает в себя техническую поддержку, консультации и рекомендации по использованию продукта в соответствии с требованиями заказчика.










\end{document}
