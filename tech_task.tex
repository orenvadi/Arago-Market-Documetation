% !TEX TS-program = pdflatex
% !TEX encoding = UTF-8 Unicode

% This is a simple template for a LaTeX document using the "article" class.
% See "book", "report", "letter" for other types of document.

\documentclass[12pt]{article} % use larger type; default would be 10pt
\usepackage[fontsize=12pt]{fontsize}
\renewcommand{\baselinestretch}{1.5} % linespacing=1.5
\usepackage[T2A]{fontenc} % кодировка
\usepackage{fontspec} % for changing fonts 
% \setmainfont{Times New Roman}
\setmainfont{Minion Pro}
% \setmainfont{Roboto}
% \usepackage[utf8]{inputenc} % set input encoding (not needed with XeLaTeX)
\usepackage[english]{babel} % for cyrillic letters support
\usepackage[parfill]{parskip} % remove space vefore parapraph
\setlength{\parskip}{0pt} % additional skip between paragraphs


%%% PAGE DIMENSIONS
\usepackage{geometry} % to change the page dimensions
\geometry{a4paper} % or letterpaper (US) or a5paper or....
\geometry{left=30mm,right=20mm,top=15mm, bottom=15mm} % for example, change the margins to 2 inches all round
% \graphicspath{{./images/}}
% \geometry{landscape} % set up the page for landscape
%   read geometry.pdf for detailed page layout information

\usepackage{graphicx} % support the \includegraphics command and options

% \usepackage[parfill]{parskip} % Activate to begin paragraphs with an empty line rather than an indent

%%% PACKAGES
\usepackage{blindtext}
\usepackage{booktabs} % for much better looking tables
\usepackage{multirow}
\usepackage{array} % for better arrays (eg matrices) in maths
\usepackage{paralist} % very flexible & customisable lists (eg. enumerate/itemize, etc.)
\usepackage{verbatim} % adds environment for commenting out blocks of text & for better verbatim
\usepackage{subfig} % make it possible to include more than one captioned figure/table in a single float
\usepackage{cmap} % search in pdf
\usepackage{longtable} % for long tables 
\usepackage{lscape}
\usepackage{hyperref} % for hyperlinks
\usepackage{listings} % for code listings
% These packages are all incorporated in the memoir class to one degree or another...

%% Useful packages
\usepackage[colorinlistoftodos]{todonotes}
% \usepackage[colorlinks=true, allcolors=blue]{hyperref}



\usepackage{amsmath,amsthm,amsfonts,amssymb,amscd, fancyhdr, color, comment, graphicx, environ}
\usepackage{float}
\usepackage{mathrsfs}
\usepackage[math-style=ISO]{unicode-math}
\setmathfont{TeX Gyre Termes Math}
\setmonofont{Hack Nerd Font Mono}
\usepackage{lastpage}
\usepackage[dvipsnames]{xcolor}
% \usepackage[framemethod=TikZ]{mdframed}
\usepackage{indentfirst}
\usepackage{thmtools}
\usepackage{shadethm}
\usepackage{setspace}


\definecolor{codegreen}{rgb}{0,0.6,0}
\definecolor{codegray}{rgb}{0.5,0.5,0.5}
\definecolor{codepurple}{rgb}{0.58,0,0.82}
\definecolor{backcolour}{rgb}{0.95,0.95,0.92}


%%% HEADERS & FOOTERS
\usepackage{fancyhdr} % This should be set AFTER setting up the page geometry
\pagestyle{fancy} % options: empty , plain , fancy
\renewcommand{\headrulewidth}{0pt} % customise the layout...
\lhead{}\chead{}\rhead{}
\lfoot{}\cfoot{\thepage}\rfoot{}

%%% SECTION TITLE APPEARANCE
\usepackage{titlesec}
\usepackage{sectsty}
\sectionfont{\centering}
\subsectionfont{\centering}

\usepackage{enumitem}
\setlist{nolistsep}

\hypersetup{
    colorlinks=true,
    linkcolor=black,
    filecolor=magenta,      
    urlcolor=blue,
    pdftitle={Overleaf Example},
    pdfpagemode=FullScreen,
    }
\urlstyle{same}




\lstdefinestyle{mystyle}{
    extendedchars=\true,
    inputencoding=utf8x,
    backgroundcolor=\color{backcolour},   
    commentstyle=\color{codegreen},
    keywordstyle=\color{magenta},
    numberstyle=\tiny\color{codegray},
    stringstyle=\color{codepurple},
    basicstyle=\ttfamily\footnotesize,
    breakatwhitespace=false,         
    breaklines=true,                 
    captionpos=b,                    
    keepspaces=true,                 
    numbers=left,                    
    numbersep=5pt,                  
    showspaces=false,                
    showstringspaces=false,
    showtabs=false,                  
    tabsize=2
}

\lstset{style=mystyle}




% \allsectionsfont{\sffamily\mdseries\upshape} % (See the fntguide.pdf for font help)
% (This matches ConTeXt defaults)

%%% ToC (table of contents) APPEARANCE
% \usepackage[nottoc,notlof,notlot]{tocbibind} % Put the bibliography in the ToC
% \usepackage[titles,subfigure]{tocloft} % Alter the style of the Table of Contents
% \renewcommand{\cftsecfont}{\rmfamily\mdseries\upshape}
% \renewcommand{\cftsecpagefont}{\rmfamily\mdseries\upshape} % No bold!
\makeatletter
\renewcommand{\fnum@figure}{Picture. \thefigure}
\makeatother



\renewcommand\lstlistingname{Algorithm}
\renewcommand\lstlistlistingname{Algorithms}
\def\lstlistingautorefname{Alg.}

\title{Market place Course Work}

\author{Baktybekov.N, Manurov.A, Seitbekov.E}

\begin{document}

\maketitle

\newpage
\tableofcontents{}
\setcounter{page}{1}

\newpage
\addcontentsline{toc}{section}{Введение}

% Введение.
% 1. Основание для разработки.
% 2. Назначение разработки.
% 3. Требования к программе или программному
% изделию.
% 4. Требования к программной документации.
% 5. Технико-экономическое обоснование.
% 6. Стадии и этапы разработки.
% 7. Порядок контроля и приемки.
% 8. Приложения.

\section*{Введение}

Данное техническое задание разработано для создания программного продукта - веб-приложения для онлайн, просмотра и размещения объявлений. Основанием для разработки является необходимость упрощения процесса поиска товаров услуг, а также повышения удобства и доступности для клиентов.

Назначение данной разработки заключается в создании функционального и удобного веб-приложения, которое позволит пользователям легко и быстро просматривать и размещать товары услуги, а также отслеживать свои объявления и получать информацию о других.

Для достижения поставленной цели необходимо выполнение следующих требований к программе: создание удобного и интуитивно понятного интерфейса, обеспечение безопасности персональных данных пользователей, интеграция с сервисами рекламных услуг и обеспечение высокой производительности и доступности приложения.

Техническое задание также устанавливает требования к программной документации, а также включает технико-экономическое обоснование, стадии и этапы разработки, порядок контроля и приемки, а также приложения, необходимые для успешной реализации проекта.









\newpage
\section{Основание для разработки.}



Основанием для разработки проекта является потребность нашей компании в создании электронной платформы для продажи товаров в Интернете. В настоящее время все больше потребителей предпочитают совершать покупки в Интернете, что делает разработку собственного интернет-магазина необходимым шагом для нашей компании в укреплении позиций на рынке и увеличении объемов продаж. Кроме того, создание собственного интернет-магазина позволит нам расширить географию нашей деятельности и увеличить количество клиентов, что в свою очередь приведет к увеличению прибыли.








% \newpage
\section{Назначение разработки.}



Назначение разработки данного проекта заключается в создании программного обеспечения, которое будет обеспечивать удобство и комфорт для пользователей при просмотре товаров и услуг на сайте. Разработка программы должна способствовать увеличению продаж, удобству навигации по сайту и повышению уровня сервиса. Также важным назначением разработки является создание функционального и надежного программного продукта, который будет соответствовать всем требованиям заказчика и пользователей.







% \newpage
\section{Требования к программе или программному изделию.}





Требования к программе или программному изделию:

    \textbf{Функциональность:}

    1.1. Возможность регистрации пользователей и создания личного кабинета;
    
    1.2. Возможность добавления, редактирования и удаления товаров и услуг;
    
    1.3. Возможность просмотра информации о товарах и услугах, включая фотографии и описания;
    
    1.4. Возможность оформления оплаты рекламы товаров и услуг;
    
    1.5. Возможность просмотра избранных объявлений и состояния собственных объявлений;
    
    1.6. Возможность обратной связи с администрацией сайта.

    \textbf{Безопасность:}
    
    2.1. Защита личных данных пользователей;
    
    2.2. Защита от SQL-инъекций и XSS-атак.

    \textbf{Скорость и производительность:}
    
    3.1. Быстрая загрузка страниц сайта;
    
    3.2. Высокая скорость обработки запросов;
    
    3.3. Поддержка большого количества пользователей одновременно.

    \textbf{Кросс-платформенность:}
    
    4.1. Совместимость с различными браузерами и операционными системами;
    
    4.2. Адаптивный дизайн для разных разрешений экранов мобильных устройств.

    \textbf{Надежность:}
    
    5.1. Бесперебойная работа в течение всего времени доступности для пользователей;
    
    5.2. Предусмотрен механизм аварийного восстановления и быстрой реакции на проблемы.

    \textbf{Удобство использования:}
    
    6.1. Простой и понятный интерфейс;
    
    6.2. Легкий поиск и отслеживание товаров и услуг;
    
    6.3. Современный дизайн, соответствующий последним трендам в веб-дизайне.

    \textbf{Поддержка:}
    
    7.1. Предусмотрен механизм технической поддержки для пользователей, 

    позволяющий быстро решать проблемы и отвечать на вопросы;
    
    7.2. Поддержка может быть реализована через электронную почту, онлайн-чат или телефон.









% \newpage
\section{Требования к программной документации.}




Требования к программной документации для данного проекта включают следующие пункты:

\begin{compactitem}

    \item    Техническое описание - должно содержать подробное описание функциональности, архитектуры и основных технологий, используемых в проекте.
    \item    Руководство пользователя - должно содержать подробное описание того, как пользоваться программным продуктом, включая инструкции по установке и настройке.
    \item    Руководство системного администратора - должно содержать описание процедур установки, настройки, администрирования и мониторинга программного продукта.
    % \item    Документация по API - должна содержать описание интерфейсов программного продукта, которые могут использоваться другими приложениями для интеграции.
    \item    Исходный код - должен быть документирован и комментирован для облегчения понимания его работы и изменения в будущем.
    \item    Тестовая документация - должна содержать описание тестовых сценариев, результаты тестирования и инструкции по проведению тестирования.
    \item    Документация по обслуживанию и поддержке - должна содержать инструкции по устранению возможных проблем и ответы на часто задаваемые вопросы.

\end{compactitem}

Все документы должны быть написаны на понятном языке, соответствовать установленным стандартам и обеспечивать достаточный уровень информации для успешной эксплуатации и поддержки программного продукта.









% \newpage
\section{Технико-экономическое обоснование.}






Технико-экономическое обоснование проекта является необходимым для определения его целесообразности и экономической эффективности. Наш проект представляет собой разработку программного обеспечения для автоматизации бизнес-процессов в компании.

Предполагается, что внедрение программы позволит значительно сократить затраты на ручной труд, сократить время на выполнение задач и увеличить эффективность работы компании в целом. Кроме того, автоматизация бизнес-процессов повысит качество работы и сократит вероятность ошибок.

Для реализации проекта необходимо закупить оборудование и программное обеспечение, а также оплатить труд программистов и тестировщиков. По результатам проведенных расчетов, общая стоимость проекта составляет X тысяч долларов США.

Ожидаемый доход от реализации проекта составляет Y тысяч долларов США. Окупаемость проекта составит Z месяцев. В перспективе, внедрение программы позволит значительно сократить затраты на ручной труд и увеличить доходы компании.

Таким образом, технико-экономическое обоснование проекта показывает его целесообразность и экономическую эффективность, а также является основой для принятия решения о дальнейшей реализации проекта.







% \newpage
\section{Стадии и этапы разработки.}






Стадии и этапы разработки проекта должны включать следующие этапы:

    \textbf{Планирование проекта:}
    
    \textit{Срок: 1 неделя}

    \begin{enumerate}

    \item    Определение целей и задач проекта;
    \item    Составление плана работ и графика выполнения;
    \item    Определение необходимых ресурсов.

    \end{enumerate}
    

    \textbf{Анализ и проектирование:}
    
    \textit{Срок: 1 неделя}

    \begin{enumerate}
            
    \item    Сбор и анализ требований к программе;
    \item    Проектирование архитектуры программного продукта;
    \item    Создание технического задания на разработку программы.
        
    \end{enumerate}

    \textbf{Разработка:}
    
    \textit{Срок: 4 недели}
    
    \begin{enumerate}

    \item    Создание программного кода;
    \item    Тестирование программного продукта;
    \item    Устранение обнаруженных ошибок и недоработок.
        
    \end{enumerate}

    \textbf{Внедрение:}
    
    \textit{Срок: 1 неделя}

    \begin{enumerate}

    \item    Подготовка к установке программы на рабочие места;
    \item    Установка программы на рабочие места;
    \item    Настройка программы и ввод данных.

    \end{enumerate}
    
    \textbf{Эксплуатация и поддержка:}
    
    \textit{Срок: 1 неделя}

    \begin{enumerate}
            
    \item    Поддержка пользователей;
    \item    Проведение работ по сопровождению программы;
    \item    Внесение изменений и улучшений в программный продукт.
        
    \end{enumerate}

Каждый этап должен быть осуществлен с соблюдением определенных этапов и процедур, и все этапы должны быть завершены до сдачи программного продукта заказчику.






\newpage
\section{Порядок контроля и приемки Приложения.}







Порядок контроля и приемки включает в себя следующие шаги:

    
    \textit{Срок: 2 недели}
\begin{enumerate}
    \item    Предварительная проверка – на этом этапе проверяются программные модули и компоненты на соответствие требованиям, а также на возможные ошибки и неполадки.
    \item    Тестирование – проводятся функциональные и интеграционные тесты, которые позволяют проверить работу программного продукта в различных условиях и на различных платформах.
    \item    Оценка качества – на этом этапе оценивается качество программного продукта по ряду критериев, таких как функциональность, удобство использования, быстродействие и т.д.
    \item    Приемка – после успешного прохождения всех проверок и тестов, производится приемка программного продукта. По результатам приемки составляется акт, в котором указывается, что продукт соответствует требованиям технического задания и может быть введен в эксплуатацию.
    \item    Гарантийное обслуживание – после ввода в эксплуатацию, проводится гарантийное обслуживание программного продукта, которое включает в себя устранение возможных ошибок и неполадок, а также обновление продукта до новых версий и выпусков.
    \item    Поддержка – после окончания гарантийного обслуживания, предоставляется услуга по поддержке программного продукта, которая включает в себя техническую поддержку, консультации и рекомендации по использованию продукта в соответствии с требованиями заказчика.

\end{enumerate}









\end{document}
